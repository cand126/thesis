\thispagestyle{plain}
\topskip0pt
\vspace*{\fill}
    \begin{center}
        \Large
        \textbf{Abstract}
    \end{center}
    
    \vspace{1cm}
    
    In a blockchain system, multiple nodes distribute transactions and blocks simultaneously. Therefore, the behaviors of nodes and the mining processes are complex and difficult to be identified. In this thesis, we present a visualization tool which examines the influences of the network delay and mining strategies in a blockchain system which employs proof-of-work as the consensus protocol. Our approach is based on the simulation of a blockchain system, which is based on a multi-agent system. The data sent between the nodes are monitored and recorded by the watchdog, and then they are sent to the visualizer. As a result, the visualizer can provide a visualization of internal blockchain processes in real-time. We provide scenarios to demonstrate the potential applications of the visualization tool. Consequently, it proves that the visualization tool is suitable for helping researchers to analyze the mining processes step by step correctly. \\
    
    \textbf{Keywords}: \textit{blockchain; visualization; mining strategy; network delay; proof-of-work}
\vspace*{\fill}

