\section{Visualization of Bitcoin}

Most of the visualization tools of Bitcoin are focused on the analysis of the relationships between transactions in time series. Giuseppe Di Battista et al. \cite{Battista2015} provided a visualization tool called BitConeView that analyzed the flows of bitcoins in the transactions by timestamps. It can be used to analyze the patterns of the flows of bitcoins and locate tainted transactions. To prevent the criminal activities and illegal behaviors that abuse the services of Bitcoin, Hiroki Kuzuno et al. \cite{Kuzuno2017} proposed an analyzing system which managed the data and statistics of blockchains for the usages of law enforcement investigation and training. Dan McGinn et al. \cite{McGinn2016} focused on visualizing transactions of Bitcoin from the top-down viewpoint. The visualization of the large scale of data provided domain experts and the general public a tool to analyze and identify the transaction patterns with big data visualization methods. The visual analytics tool that was proposed by Loïs Saublet \cite{Saublet2015} enabled economists to analyze the metrics and actors in Bitcoin and provided good user experience by working together with end users.

There are some tools and applications that provide visualization of the static status of transactions and blocks in Bitcoin. Tri A. Sundara et al. \cite{Sundara2017} summarized 8 visualization applications and compared their visualization methods and technologies.

In short conclusion, the goals of the above tools and applications were to analyze the patterns and relationships of transactions and blocks. Thus, they tended to provide the static information and statistics instead of the dynamic mining processes that happen in a blockchain system. To focus on the dynamic mining activities, our tool provides a clear, real-time visualization of mining processes that cannot be achieved by the above analysis tools.

\section{Analysis of Proof-of-Work}

To analyze the network conditions of the proof-of-work based blockchain system, Amitai Porat et al. \cite{Porat} provided an application that was based on Ethereum platform. It simulated the asynchronous mining activities and network latency and provided a real-time analysis of the blockchain system.

The main difference to our work is that we assume that miners have individual mining strategies instead of solving the same puzzles asynchronously. Therefore, there are miners who are able to solve the puzzles faster than the others under the same computing power, but they may suffer lower mining rewards. The combination of different mining strategies and delays of networks makes the visualization of the blockchain system dynamic and interesting, and it is suitable for researchers to find the influences of them on the blockchain system.

\section{Our approach}

There is a former project called “Kooperative Music Box” \cite{musicbox}, which was developed by Fraunhofer Blockchain-Labor. It used shapes and colors to distinguish different blocks and transactions which were generated by different nodes. The basic visualization ideas of our project are inspired by this application.

Our project aims to visualize all the dynamic mining processes that could happen in a blockchain system, with the factors of the different mining strategies and the delays of networks. This distinguishes our project to all the projects mentioned before. Therefore, our project is worthwhile as a contribution and trial to the blockchain research area.
