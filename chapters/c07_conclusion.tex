\section{Summary}

This thesis proposes the first trial to visualize the complex and dynamic mining processes in the blockchain system. The visualization is based on the simulation of the blockchain system which is powered by the multi-agent system. With the watchdog which monitors the data in the blockchain system, the visualizer can update the visualization in real-time after it receives the data from the watchdog. The visualization of blockchain processes clearly indicates the mining activities step by step. 

The visualization application helps researchers to conduct experiments and focus on the mining activities which are influenced by the network delay and the mining strategy in the blockchain system. By providing a configuration file which defines all the properties and parameters in the blockchain system, the same mining processes can be replayed by the visualization application. With the discussions of the scenarios, it proves the potential applications of the visualization application.

The process of this thesis begins with designing the visualization. To build a simple blockchain visualization, we decided to use colors to represent the source of the block. We did not consider different sizes of blocks, which are proved as a good visualization in other visualization applications such as Bitcoincity and Bitbonkers because it can help users to understand the rewards from mining. For transactions, we visualize the status of transaction pools instead of visualizing the details of transactions. It was a good decision because it does not obscure our focus on the blockchain visualization. The deisgn style is inspired by material design \cite{material}, a light yet beautiful design which can satisfy the requirement of simple visualization design.

After that, we need to decide the base of visualization, i.e., the blockchain system. The selection of the multi-agent system rather than using blockchain platforms such as Ethereum is a right choice in the implementation. Because we need different mining strategies for each miner, it cannot be achieved in these blockchain platforms. As a result, the only way is to implement a simulation of a blockchain system by ourselves.

Finally, we decided that the factors of visualization are network delay and mining strategies. Network delay is simulated by the scheduling functionalities of the program, which is a reasonable way. Mining strategies do not exist in the real blockchain systems, so we decided the parameters by ourselves. We emphasized the computing power and the selection of transactions in the mining activities. The weakness of this model is that the attacks cannot be identified in the visualization. Designing a blockchain visualization which can prevent attacks will be the next step.

\section{Future Work}

In the future, we plan to do the following improvements to the visualization application.

\begin{itemize}
    \item \textbf{Add different consensus protocols} \\
        The consensus protocol can be extended to include other algorithms, such as proof-of-stake. Therefore, researchers can switch between different consensus protocols to compare different mining behaviors.
    \item \textbf{Add malicious nodes} \\
        To simulate the attacks on the blockchain system, malicious nodes should be included in the application. Thus, we need to improve the algorithm of consensus protocols to validate and detect malicious nodes.
    \item \textbf{Dynamic nodes} \\
        In a peer-to-peer network, nodes can jump in and out at any time. Ensuring the dynamic creations and deletions while the visualization is running could be a good feature.
    \item \textbf{Merge of forks} \\
        To solve the merge problem in the second scenario, the naive solution is to provide a randomness to the parameters in the blockchain system. For example, the duration of mining time should be randomness between a range because solving a puzzle cannot be predicted in the real blockchain system.
    \item \textbf{Confirmation of blocks} \\
        The longest blockchain sometimes changes in the visualization. It can be improved to implement the confirmation of blocks. For example, a block is confirmed if it is 6 blocks deep and on the longest blockchain in Bitcoin.
    \item \textbf{Different sizes of blocks} \\
        In the visualization, the sizes of blocks are the same even if their values are different. Many visualization tools such as Bitcoincity and Bitbonkers distinguish different sizes of blocks. Therefore, it is a good feature to let the users get more information about the reward.
\end{itemize}
