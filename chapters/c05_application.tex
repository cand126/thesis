In this chapter, we introduce the usage of the visualization application. It begins with a simple example to demonstrate the useful features of the visualization application. Finally, the explanation of the configuration files is provided.

\section{Flow}
\label{sec:flow}

The general flow of the visualization application is guided here. In the beginning, the blockchain system is empty, and the user can either upload a configuration file or add nodes manually (Figure \ref{fig:start of the application}). The introduction starts with creating nodes and publishing transactions manually, as it is clearer to begin with a simple example.

\begin{figure}[htb]
    \centering
    \includegraphics[width=\textwidth]{application_start}
    \caption{Start of the Application.}
    \label{fig:start of the application}
\end{figure}

By clicking the top right button in the navigation bar, the user will be redirected to the settings page, as Figure \ref{fig:settings page} shows.

\begin{figure}[htb]
    \centering
    \includegraphics[width=\textwidth]{application_settings}
    \caption{Settings Page.}
    \label{fig:settings page}
\end{figure}

The settings page contains all the nodes that exist on the blockchain network. However, the blockchain system is empty currently, so the settings page is empty. To add a node, the user can click ``ADD A MINER'' button or ``ADD A NONMINER'' button. In this example, we add three miners and two nonminers.

\begin{figure}[htb]
    \centering
    \includegraphics[width=\textwidth]{application_settings_tg}
    \caption{Settings of Transaction Generator.}
    \label{fig:settings of transaction generator}
\end{figure}

After creating nodes, the settings page contains three miners and two nonminers now. Additionally, the transaction generator is added to the blockchain system automatically. In Figure \ref{fig:settings of transaction generator}, the properties of the transaction generator are displayed in the most top block. The properties contain the ID and the network delay to its neighbors.

By scrolling down the settings page, the miners that exist in the blockchain system are displayed first. The parameters of the miners can be divided into three parts, as Figure \ref{fig:settings of miner} shows.

\clearpage

\begin{figure}[htb]
    \centering
    \includegraphics[width=\textwidth]{application_settings_m}
    \caption{Settings of Miner.}
    \label{fig:settings of miner}
\end{figure}

\begin{itemize}
    \item \textbf{Information} \\
        The properties of the miner, including the ID, the name, and the represented color.
    \item \textbf{Network Delay} \\
        It defines the periods of delay to the neighbors.
    \item \textbf{Mining Strategy} \\
        This part contains the four parameters that are used for the mining strategy.
\end{itemize}

\begin{figure}[htb]
    \centering
    \includegraphics[width=\textwidth]{application_settings_nm}
    \caption{Settings of Nonminer.}
    \label{fig:settings of nonminer}
\end{figure}

The last part of the settings page is for the nonminers. Because nonminers do not mine a block by themselves, the represented color and the parameters of mining strategy are missing here, as Figure \ref{fig:settings of nonminer} shows.

The user can give miners and nonminers nicknames such as Alice, Bob, etc., to make the relationship between the nodes more understandable. The nicknames may not be unique in the blockchain system for the reason that the nodes are recognized by their unique IDs. Moreover, the parameters are updated automatically while the user is typing. All the parameters are generated randomly at first, so it is better to check and set each parameter manually.

In this example, we set Alice (red color), Bob (blue color), and Charlie (green color) as miners, and David and Eva as nonminers. For the mining strategy, we set the following three parameters to the same number for all miners.

\begin{itemize}
    \item Mining Time (Seconds): 1
    \item Number of Mined Transactions: 1
    \item Maximum Number of Pending Transactions: 10
\end{itemize}

For the parameters of \textit{minimum value of transactions}, we set 1 for Alice, 5 for Bob, and 9 for Charlie. Therefore, Alice is expected to mine blocks faster than Bob and Charlie. For the network delay, we set all the parameters to 1 second.

\begin{figure}[htb]
    \centering
    \includegraphics[width=\textwidth]{application_index}
    \caption{Initial status.}
    \label{fig:initial status}
\end{figure}

After finishing the configuration, the user can see the initial status of the blockchain system like Figure \ref{fig:initial status}. Each row represents the visualization of a miner or nonminer. A miner has a colorful head and two progress bars which display the status of the transaction pool and the mining activity. At the bottom of each row, it is the main area for visualizing the blockchain data structure. In the beginning, the grey block represents the genesis block.

\begin{figure}[htb]
    \centering
    \includegraphics[width=\textwidth]{application_demo1}
    \caption{Alice mined a block.}
    \label{fig:alice mined a block}
\end{figure}

The user can publish a transaction at the bottom area of the screen. By entering a number of the reward, a transaction will be generated by the transaction generator and published. In the first step, we publish a transaction which reward is 1. Hence, it is expected that only Alice will mine a block because the \textit{minimum value of the transactions} is 1 for Alice, and the necessary \textit{number of mined transactions} is also 1. The result is showed in Figure \ref{fig:alice mined a block}. Moreover, now Alice's total reward is 1 because she added a block to the longest blockchain successfully.

\begin{figure}[htb]
    \centering
    \includegraphics[width=\textwidth]{application_demo2}
    \caption{Alice and Bob mined a block simultaneously.}
    \label{fig:alice and bob mined a block simultaneously}
\end{figure}

Figure \ref{fig:alice and bob mined a block simultaneously} demonstrates that Alice and Bob competed with each other by adding a block at the same time. Because we published a transaction with the reward of 5, Alice and Bob both decided to mine a block according to their mining strategies, i.e, the reward of the transaction is not less than their \textit{minimum value of the transactions}. As a result, the fork happened for the reason of simultaneous mining activities.

\begin{figure}[htb]
    \centering
    \includegraphics[width=\textwidth]{application_demo3}
    \caption{Charlie mined a block.}
    \label{fig:charlie mined a block}
\end{figure}

In the next step, Figure \ref{fig:charlie mined a block} shows the influence of the mining strategy to the mining processes. We changed the parameters of the mining strategy at first. Thus, Alice's and Bob's parameters of \textit{number of mined transactions} were both changed to 2. After that, a transaction with the reward of 9 was published. The result was that Charlie mined a block for the reason that it satisfied Charlie's mining strategy, i.e, the \textit{number of mined transactions} is only 1 for Charlie. On the other hand, Alice and Bob did not mine a block because of the insufficient number of transactions.

\begin{figure}[htb]
    \centering
    \includegraphics[width=\textwidth]{application_demo4}
    \caption{Charlie mined another block.}
    \label{fig:charlie mined another block}
\end{figure}

Figure \ref{fig:charlie mined another block} shows the process of selecting candidate transactions. We published a transaction with the reward of 1. The result was that Charlie mined a block after about 8 seconds due to the process of selecting candidate transactions. Normally, a miner starts to select candidate transactions in every second, and the privileges of transactions that are not selected were added by 1 each time. Consequently, the value of the published transaction was becoming 9 after 8 seconds, and it satisfied Charlie's mining strategy, i.e., the \textit{minimum value of the transactions}.

\begin{figure}[htb]
    \centering
    \includegraphics[width=\textwidth]{application_demo5}
    \caption{Overflow of transaction pools.}
    \label{fig:overflow of transaction pools}
\end{figure}

Finally, Figure \ref{fig:overflow of transaction pools} shows the overflow of the transaction pools. First, the parameters of \textit{minimum value of the transactions} were all set to 15 for the three miners, and then we published multiple transactions with the reward of 1. Because of a large number of transactions with low reward is published, the transaction pools became full in a short time, and it was displayed in the progress bars for transaction pools. As mentioned in the mining algorithm, all the three miners are forced to mine a block due to the overflow of the transaction pools, i.e., the number of pending transactions is larger than 10 in this example. The result can be checked in Figure \ref{fig:alice, bob, and charlie were forced to mine}, and it is the same as the expectation that all miners mine blocks simultaneously.

\begin{figure}[htb]
    \centering
    \includegraphics[width=\textwidth]{application_demo6}
    \caption{Alice, Bob, and Charlie were forced to mine.}
    \label{fig:alice, bob, and charlie were forced to mine}
\end{figure}

\clearpage

During the mining processes, the following auxiliary information in the visualization is also helpful and dynamic.

\begin{itemize}
    \item \textbf{Progress bars for the transaction pool} \\
        It displays the level of capacity of the transaction pool.
    \item \textbf{Progress bars for the mining status} \\
        It presents the mining activities in real-time, i.e., how much time is remaining for solving a puzzle.
    \item \textbf{Number of total rewards} \\
        It displays the calculation of the total reward that belongs to the miner.
    \item \textbf{Number of forks} \\
        On the top of each fork, there is a number that represents the number of forks.
\end{itemize}

Additionally, the user can drag the visualization area to move the blockchain data structure, and zoom in and out by clicking the plus and minus buttons on the bottom right side.

\section{Configuration Files}

To replay the same visualization of blockchain processes, we can define the configuration of the blockchain system in a file and upload it to the application as in Figure \ref{fig:start of the application}. The configuration file is composed of the following three parts.

\begin{itemize}
    \item the properties and the parameters of the mining strategy of nodes
    \item the network delay between each node
    \item the transactions that will be published through the network
\end{itemize}

We provide a sample configuration file in Listing \ref{lst:sample of configuration file}. The file is in JSON format, and the file is used to initialize the blockchain system in the beginning.

First, for nodes, it is important to define a unique transaction generator at the beginning of the file, and then it is followed by miners and nonminers. The required properties for the transaction generator, miners, and nonminers are provided as the following.

\clearpage

\begin{enumerate}
    \item \textbf{Transaction Generator}
        \begin{itemize}
            \item id
            \item type
        \end{itemize}
    \item \textbf{Miner}
        \begin{itemize}
            \item id
            \item type
            \item color
            \item name
            \item miningTime
            \item minValue
            \item mineNumber
            \item maxPending
        \end{itemize}
    \item \textbf{Nonminer}
        \begin{itemize}
            \item id
            \item type
            \item name
        \end{itemize}
\end{enumerate}

Thd \textit{id} of nodes can be a string with arbitrary length, and here we use 8 characters and digits to represent the identifier. The \textit{type} distinguishes different nodes, and the valid values are \textit{generator}, \textit{miner}, and \textit{nonminer}. The \textit{name} of miners and nonminers are nicknames for them. Miners also have unique \textit{colors} to differentiate blocks in the visualization. The parameters of \textit{miningTime}, \textit{minValue}, \textit{mineNumber}, and \textit{maxPending} are from the mining strategy, which are \textit{mining time}, \textit{minimum value of transactions}, \textit{number of mined transactions}, \textit{minimum number of pending transactions} respectively.

The second part of the configuration file is a list of delay between each node. Each node has a block that begins with the \textit{id}, and then it is followed by a list of the neighbors and the amount of delay. As mentioned before, the transaction generator only has miners as neighbors, while miners and nonminers connect with each other. Moreover, the transaction generator is not in the neighbors of miners and nonminers because it is meaningless to send blocks to the transaction generator.

In the last part, it contains a list of transactions with the number of rewards and the delay after the starting of the blockchain system, i.e., we can arrange the transactions to be published in a sequence.

